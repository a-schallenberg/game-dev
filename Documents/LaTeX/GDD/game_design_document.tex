\documentclass[paper=A4,pagesize=auto,12pt,headinclude=true,footinclude=true,BCOR=0mm,DIV=calc]{scrartcl}

\usepackage[T1]{fontenc}
\usepackage[utf8]{inputenc}
\usepackage[ngerman]{babel}
\usepackage{lmodern}
\usepackage[hidelinks]{hyperref}
\usepackage{setspace}
\usepackage[left=25mm, right=25mm, top=25mm, bottom=25mm]{geometry}
\usepackage[backend=biber, style=ieee, citestyle=ieee]{biblatex}
\usepackage[nottoc,numbib]{tocbibind}
\onehalfspacing

\bibliography{game_design_document}

% Title
\title{Game Design Document}

\author{Alexandra Zarkh, Sui Yin Zhang, Lennart Leggewie,\\ Stefan Glorch, Alexander Schallenberg}


\makeatletter
\def\@maketitle{%
	\newpage
	\null
	\vskip 2em%
	\begin{center}%
		\let \footnote \thanks
		{\Huge \textbf{\@title} \par}%
		\vskip 1.5em%
		{\Large Game Development: ``\gametitle''\par}%
		\vskip 3em%
		{\large
			\lineskip .5em%
			\begin{tabular}[t]{c}%
				\@author
			\end{tabular}\par}%
		\vskip 2em%
		{\large Hochschule Bonn-Rhein-Sieg, Fachbereich Informatik, D-53757\par}%
		\vskip 2em%
		{\large \@date}%
	\end{center}%
	\par
	\vskip 1.5em}
\makeatother


% Commands
\newcommand{\sectionspace}{
	\vspace{0.5cm}
}

\newcommand{\gametitle}{Chance of Hell}




\begin{document}
	
\begin{titlepage}
	\maketitle
\end{titlepage}


\tableofcontents
\newpage

\section{Hintergrund}\label{sec:Hintergrund}

\sectionspace
\section{Formale Elemente}\label{sec:Formale_Elemente}

\sectionspace
\subsection{Spielende (Zielgruppe)}\label{sec:Spieler}
Die spielende Person spielt einen ins Spielgeschehen aktiv eingebundenen Avatar, mit der sie zunächst überwiegend bauen, im späteren Spielverlauf zum großen Teil managen, aber immer aktiv mit der Spielwelt interagieren soll. Die Top-Down Perspektive mit dem Avatar im Mittelpunkt soll der spielenden Person die nötige nähe des Charakters vermitteln, dennoch genügend Abstand bieten, um das Umfeld im Blick zu behalten. Das Spiel wird ausschließlich im Einzelspieler-Modus spielbar sein, in welchem die spielende Person gegen Herausforderungen des Spiels antritt. Je nach eingestelltem Schwierigkeitsgrad sind diese leichter bzw. schwieriger und treten seltener bzw. häufiger auf. Dadurch entstehen teilweise und je nach Schwierigkeitsgrad kurze Spannungsmomente, die den Handlungsdruck kurzzeitig erhöhen. Dieser wird jedoch durch das ruhige Spielgeschehen außerhalb der Spannungsmomente abgeschwächt. Die Grafik, welche im Gegensatz zur Realität eine deutlich geminderte Auflösung besitzt, trägt dazu bei, dass auch Spielende im Grundschulalter leicht Abstand zum Spiel erhält.\\ % Frage: zitieren ?
Da es sich bei diesem Spiel um ein Aufbau-Strategiespiel handelt, welches recht friedlich, in gewissen Situationen dennoch anspruchsvoll ist, eignet sich das Spiel nicht für jede Altersgruppe, kann aber auch von jungen Spielenden verstanden und gespielt werden. Die in Deutschland für die Altersfreigabe von Videospielen verantwortliche Stelle ``Unterhaltungssoftware Selbstkontrolle'' (kurz USK) stuft Aufbau-Strategiespiele meistens unter USK ab 6 Jahren \cite{usk_6} ein. Durch die oben genannten Aspekte wie realtitätsdistanzierte Grafik würde dies auch für dieses Spiel eine angemessene Altersgrenze darstellt. \\ 
Auch für ältere Spielende eignet sich das Spiel. Hierbei muss jedoch beachtet werden, dass bestimmte mentale Fähigkeiten wie die Hand-Auge-Koordination in gewissem Ausmaß gefordert sind, die bei der spielenden Person vorhanden sein müssen. Sind diese Fähigkeiten bei Personen ab 6 Jahren nicht (mehr) vorhanden, ist das Spiel für diese Spielenden ungeeignet.

\sectionspace
\subsection{Gegenstand \& Ziel des Spiels}\label{sec:Gegenstand}

\sectionspace
\subsection{Abläufe}\label{sec:Ablaeufe}

\sectionspace
\subsection{Regeln}\label{sec:Regeln}

\sectionspace
\subsection{Ressourcen}\label{sec:Ressourcen}

\sectionspace
\subsection{(potentielle) Konflikte, absehbare Schwierigkeiten}\label{sec:Konflikte}

\sectionspace
\subsection{Rahmenbedingungen}\label{sec:Rahmenbedingungen}

\sectionspace
\subsection{Ergebnisse}\label{sec:Ergebnisse}
\gametitle \space ist eine eher ruhig gehaltenes Spiel

\sectionspace
\printbibliography[heading=bibnumbered, title=Referenzen]\label{sec:Referenzen}

\end{document}
