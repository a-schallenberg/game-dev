\documentclass[paper=A4,pagesize=auto,12pt,headinclude=true,footinclude=true,BCOR=0mm,DIV=calc]{scrartcl}

\usepackage[T1]{fontenc}
\usepackage[utf8]{inputenc}
\usepackage[ngerman]{babel}
\usepackage{lmodern}
\usepackage[hidelinks]{hyperref}
\usepackage{setspace}
\usepackage[left=25mm, right=25mm, top=25mm, bottom=25mm]{geometry}
\usepackage[backend=biber, style=ieee, citestyle=ieee]{biblatex}
\usepackage[nottoc,numbib]{tocbibind}
\onehalfspacing

\bibliography{game_design_document}

% Title
\title{Game Design Document}

\author{Alexandra Zarkh, Sui Yin Zhang, Lennart Leggewie,\\ Stefan Glorch, Alexander Schallenberg}


\makeatletter
\def\@maketitle{%
	\newpage
	\null
	\vskip 2em%
	\begin{center}%
		\let \footnote \thanks
		{\Huge \textbf{\@title} \par}%
		\vskip 1.5em%
		{\Large Game Development: ``\gametitle''\par}%
		\vskip 3em%
		{\large
			\lineskip .5em%
			\begin{tabular}[t]{c}%
				\@author
			\end{tabular}\par}%
		\vskip 2em%
		{\large Hochschule Bonn-Rhein-Sieg, Fachbereich Informatik, D-53757\par}%
		\vskip 2em%
		{\large \@date}%
	\end{center}%
	\par
	\vskip 1.5em}
\makeatother


% Commands
\newcommand{\sectionspace}{
	\vspace{0.5cm}
}

\newcommand{\gametitle}{Chance of Hell}




\begin{document}
	
\begin{titlepage}
	\maketitle
\end{titlepage}


\tableofcontents
\newpage

\section{Hintergrund}\label{sec:Hintergrund}

\sectionspace
\section{Formale Elemente}\label{sec:Formale_Elemente}

\sectionspace
\subsection{Spielende (Zielgruppe)}\label{sec:Spieler}
Die spielende Person spielt einen ins Spielgeschehen aktiv eingebundenen Avatar, mit der sie zunächst überwiegend bauen, im späteren Spielverlauf zum großen Teil managen, aber immer aktiv mit der Spielwelt interagieren soll. Die Top-Down Perspektive mit dem Avatar im Mittelpunkt soll der spielenden Person die nötige nähe des Charakters vermitteln, dennoch genügend Abstand bieten, um das Umfeld im Blick zu behalten. Das Spiel wird ausschließlich im Einzelspieler-Modus spielbar sein, in welchem die spielende Person gegen Herausforderungen des Spiels antritt. Je nach eingestelltem Schwierigkeitsgrad sind diese leichter bzw. schwieriger und treten seltener bzw. häufiger auf. Dadurch entstehen teilweise und je nach Schwierigkeitsgrad kurze Spannungsmomente, die den Handlungsdruck kurzzeitig erhöhen. Dieser wird jedoch durch das ruhige Spielgeschehen außerhalb der Spannungsmomente abgeschwächt. Die Grafik, welche im Gegensatz zur Realität eine deutlich geminderte Auflösung besitzt, trägt dazu bei, dass auch Spielende im Grundschulalter leicht Abstand zum Spiel erhält.\\ % Frage: zitieren ?
Da es sich bei diesem Spiel um ein Aufbau-Strategiespiel handelt, welches recht friedlich, in gewissen Situationen dennoch anspruchsvoll ist, eignet sich das Spiel nicht für jede Altersgruppe, kann aber auch von jungen Spielenden verstanden und gespielt werden. Die in Deutschland für die Altersfreigabe von Videospielen verantwortliche Stelle ``Unterhaltungssoftware Selbstkontrolle'' (kurz USK) stuft Aufbau-Strategiespiele meistens unter USK ab 6 Jahren \cite{usk_6} ein. Durch die oben genannten Aspekte wie realitätsdistanzierte Grafik würde dies auch für dieses Spiel eine angemessene Altersgrenze darstellt. \\ 
Auch für ältere Spielende eignet sich das Spiel. Hierbei muss jedoch beachtet werden, dass bestimmte mentale Fähigkeiten wie die Hand-Auge-Koordination in gewissem Ausmaß gefordert sind, die bei der spielenden Person vorhanden sein müssen. Sind diese Fähigkeiten bei Personen ab 6 Jahren nicht (mehr) vorhanden, ist das Spiel für diese Spielenden ungeeignet.

\sectionspace
\subsection{Gegenstand \& Ziel des Spiels}\label{sec:Gegenstand}

\sectionspace
\subsection{Abläufe}\label{sec:Ablaeufe}

\sectionspace
\subsection{Regeln}
Der spielenden Person ist es möglich, den Avatar in einem begrenzten Gebiet frei umherzubewegen. In diesem Gebiet ist es ihr möglich, Gebäude zu platzieren, die jeweils einen speziellen Nutzen haben oder einfach nur dekorativ sind. Das einzige Gebäude, welches von Anfang an freigeschaltet ist, ist der Workshop. Dieser dient einerseits dazu, einen Überblick über die baubaren Gebäude zu bieten und anderseits das Freischalten von neuen Gebäuden zu ermöglichen. \\
Das Platzieren der Gebäude erfolgt nicht direkt. Vielmehr stellt die spielende Person ein Grundgerüst im Workshop her, was dann vor dem Avatar in der Spielwelt platziert wird. Das einzige Gebäude, was direkt platziert werden kann, ist der Workshop selbst, da er mit genügend Ressourcen im Inventar des Avatars hergestellt werden kann.
Ist im Fortschrittsbaum des Workshops eine bestimmte Stufe erreicht worden, so wird es möglich sein, eine Verbesserung des Workshops zu einem Forschungslabor freizuschalten, die wiederum das Erforschen von fortgeschritteneren Gebäuden erlaubt. \\
Die im weiteren Spielverlauf auftauchenden Nicht-Spieler-Charakter (NPCs) müssen durch Annehmlichkeiten wie Dekoration und dem vorsorglichen Schutz vor auftretenden negativen Spielereignissen zufrieden gehalten werden. Sinkt die Zufriedenheit eines NPCs unter den Grenzwert von 30 \%, besteht die Gefahr, dass der NPC sich ein neues Zuhause sucht und nicht länger Bewohner der Siedlung ist. \\ % (30 % noch diskutierbar, bislang arbiträrer Wert)
Die Zufriedenheit eines NPCs hat des Weiteren auch Auswirkung auf die Arbeitseffizienz. Bei geringerer Effizienz wird auch eine geringere Anzahl an Ressourcen generiert.


\sectionspace
\subsection{(potentielle) Konflikte, absehbare Schwierigkeiten}

\sectionspace
\subsection{Ressourcen}\label{sec:Ressourcen}
<hier noch ein bisschen Text einfügen>

\subsubsection{Materialien}
Die Start Ressourcen in diesem Spiel werden Holz und ein Seil sein welches man am Start des Spieles erhält. Weiteres Holz erhält man durch das fällen von Bäumen oder durch eine Holzfäller Hütte, die jedoch erst später im spiel errichtet werden kann. Zusätzlich gibt es auch noch Steine und Eisen, welche man aus Minen bekommt die man jedoch zuerst erbauen muss.\\
Über die ganze Welt sind auch Blumen verteilt die zur Erstellung von Farben benutzt werden können.\\
Außerdem findet man auch verschiedene Gräser von denen man Samen zum anpflanzen von Gemüse erhalten kann. 

\subsubsection{Gebäude}
Als erstes gibt es einen Workshop indem man mit den gesammelten Materialien verschiedene Gebäude oder weiteres zusammensetzen kann.\\
Unter anderem ein Lagerhaus welches das Inventar des Spielenden erweitert. Außerdem gibt es wie oben erwähnt eine Holzfäller Hütte die man in der Nähe eines Waldes platziert. Um das gesammelte Holz zu verarbeiten, wird es auch ein Sägewerk geben.\\
Ein weiteres Gebäude wäre die Mine die für das erhalten von Rohstoffen wie Eisen und Steinen benötigt wird.\\
Es gibt auch noch eine Farm die zum anbauen von Gemüse benutzt werden kann.

\subsubsection{Dekorationen}
Um seine Insel zu verschönern wird es auch Brücken, Lampen, Wege und ähnlichen zum zusammensetzen im Workshop zur Verfügung gestellt.\\
Dazu gibt es auch noch Farben die zur Färbung von Gebäuden verwendet werden.

\subsubsection{Katastrophen Kontrolle}
Damit die spielende Person sein Dorf vor verschiedenen Katastrophen schützen kann, gibt es Konstruktionen wie einen Blitzableiter der die Gebäude in einem bestimmten Bereich vor Blitzen schützt.

\subsubsection{Bewohner (NPC's)}
Die Bewohner des Dorfes haben zwar kein Leben, jedoch hat jeder Bewohner eine Zufriedenheit die der Spieler aufrechterhalten muss, da die Bewohner sonst die Insel verlassen.\\
Diese Zufriedenheit ergibt sich daraus, ob das Dorf sicher vor Katastrophen ist.

\sectionspace
\subsection{(potentielle) Konflikte, absehbare Schwierigkeiten}\label{sec:Konflikte}

Im Spiel können gewisse Hindernisse und Dilemmas auftreten, die zum Spielspaß beitragen sollen.\\

Am Anfang wird das größte Hindernis erstmal sein, als spielende Person selber zu überleben.
Durch das Erkunden der Umgebung kann die spielende Person Essen und andere Ressourcen finden, um den eigenen Hunger zu stillen, sowie einen warmen Ort für sich aufzubauen.\\
Wenn die spielende Person an einen Standpunkt gekommen ist, wo sie genug Erfahrung und Materialien gesammelt hat, werden nun NPC’s erscheinen, die mit dem spielende Person zusammen an diesem Ort leben wollen, die der spielende Person nun auch verwalten muss.\\
Man soll als spielende Person jetzt das eigene Überleben sowie das Überleben der NPC’s sichern, dennoch können auch Dilemmas und Katastrophen innerhalb des nun aufgebauten Dorfes entstehen/erscheinen, die durch die NPC’s entstehen können.\\
Zunächst könnte im Dorf eine Seuche ausbreiten, die nur die spielende Person aufhalten kann, in dem sie ein Gegenmittel entwickelt, welche durch das Erkunden von der Umgebung und neuen Ressourcen entdeckt/erlernt werden kann. \\
Außerdem können auch gängige Naturkatastrophen auftreten, wie z. B. eine Flut, die ein gewissen Dorfteil überschwemmt. In diesen Überschwemmten Dorfteilen müssten man die Zeit wieder investieren, um sie wiederherzustellen oder zu optimieren, damit eine zukünftige Flut keine großen Schäden hinterlassen kann.\\
Die Dorfbewohner können jedoch auch selber Schäden verursachen durch alltägliche Unfälle z. B. Brände. Als Beispiel hätten wir ein nicht erloschenes Lagerfeuer, welches zur Folge einen Waldbrand auslösen würde. Also müsste die spielende Person entweder das Lagerfeuer vorher selber auslöschen oder Bewohner eine Rolle zuweisen, die auf so welche Unfälle sich spezialisieren (z.B Feuerwache). \\
Es können mehrere Hindernisse und Katastrophen aufkommen, welche die spielende Person parallel verhindern muss. Wenn die spielende Person dieses nicht überwältigen kann oder sie ignoriert, können daraus sich noch mehr Hindernisse bilden, z.B Lebensmittelmangel führt zur Hungersnot.
\sectionspace
\subsection{Rahmenbedingungen}\label{sec:Rahmenbedingungen}
Im Folgenden wird auf die verschiedenen Rahmenbedingungen eingegangen, denen das Projekt unterliegt. Es findet dabei eine Einteilung in die Kategorien technische, juristische und zeitliche Rahmenbedingungen statt. % still room for improvement
\subsubsection{Technische Rahmenbedingungen}
Die wohl elementarste Grenze der technischen Möglichkeiten bilden die Grenzen der verwendeten Engine Unity.  Diese ist zwar durch Erweiterungen fast beliebig erweiterbar, aber etwa Erweiterungen, die das Schreiben eines Mehrspielerspiels ermöglichen, setzen ein entsprechend angepasstes Programmierkonzept voraus, welches den zeitlichen Rahmen dieses Projekts sprengen würde.
\subsubsection{Juristische Rahmenbedingungen}
Um etwaige juristische Probleme zu vermeiden, wird bei der Auswahl der verwendeten Assets, seien es Texturen, Sounds, Schriftarten, etc., darauf geachtet, dass diese eine Verwendung im Rahmen des Projekts erlauben. Bevorzugt werden diese, die entweder in der Public Domain oder aber unter einer Lizenz wie der Creative Commons \cite{cc_licenses} oder ähnlichen veröffentlicht wurden. Quellen für solche sind in diesem Fall der etwa der Unity Asset Store oder Online-Marktplätze wie itch.io. Dort, wo es zeitlich und technisch möglich ist, werden jedoch Assets aus eigener Produktion verwendet.
\subsubsection{Zeitliche Rahmenbedingungen}
Aufgrund des zeitlichen Rahmens findet eine Einteilung der geplanten Spielelemente in verschieden Prioritätskategorien statt. Gleichzeitig wird die Implementierung des Spiels möglichst modular gehalten, sodass bei eventuellem Zeitdruck Spielelemente mit niedriger Priorität ausgelassen werden können, ohne die Stabilität der restlichen Spielelemente zu gefährden.

\sectionspace
\subsection{Ergebnisse}\label{sec:Ergebnisse}
Durch die Story in ``\gametitle'' endet, indem der Teufel so beeindruckt von der Stadt ist, dass er der spielenden Person einiges der Schuld beim Teufel erlässt. Das geschieht, wenn die höchste Stufe des Rathauses erreicht wurde und die Zufriedenheit aller Bewohner im oberen viertel liegt. Damit ist die Story gewonnen, jedoch hat die spielende Person die Möglichkeit, das Spiel ohne Story fortzusetzen. Damit ist das Spiel ein Endlosspiel, dessen einzige Grenze der Platz auf der Karte ist. Ist die Karte komplett verbaut, kann der Spieler höchstens alte Gebäude abreißen und Neue bauen. Das Spiel kann weiterhin jederzeit durch Katastrophen, die nicht ausreichend bekämpft werden, verloren werden.
Das Spiel fragt also Bedingungen ab, die dazu führen können, dass die spielende Person gegen das Spiel (bzw. die Katastrophen) verliert und dass die Story durch erreichen der Ziele abgeschlossen ist. Die spielende Person kann nur in gewissen Situationen, wie die erfolgreiche Bekämpfung von Katastrophen, gegen das Spiel gewinnen. jedoch nicht so, dass das Spiel danach beendet wäre. Ebenso ist es nicht möglich gegen die Story zu verlieren, allerdings kann sie beliebig verzögert werden.


\sectionspace
\printbibliography[heading=bibnumbered, title=Referenzen]\label{sec:Referenzen}

\end{document}
