\documentclass[paper=A4,pagesize=auto,12pt,headinclude=true,footinclude=true,BCOR=0mm,DIV=calc]{scrartcl}

\usepackage[T1]{fontenc}
\usepackage[utf8]{inputenc}
\usepackage[ngerman]{babel}
\usepackage{lmodern}
\usepackage[hidelinks]{hyperref}
\usepackage{setspace}
\usepackage[left=25mm, right=25mm, top=25mm, bottom=25mm]{geometry}
\usepackage[backend=biber, style=ieee, citestyle=ieee]{biblatex}
\usepackage[nottoc,numbib]{tocbibind}
\onehalfspacing

\bibliography{game_design_document}

% Title
\title{Game Design Document}

\author{Alexandra Zarkh, Sui Yin Zhang, Lennart Leggewie,\\ Stefan Glorch, Alexander Schallenberg}


\makeatletter
\def\@maketitle{%
	\newpage
	\null
	\vskip 2em%
	\begin{center}%
		\let \footnote \thanks
		{\Huge \textbf{\@title} \par}%
		\vskip 1.5em%
		{\Large Game Development: ``Chance of Hell''\par}%
		\vskip 3em%
		{\large
			\lineskip .5em%
			\begin{tabular}[t]{c}%
				\@author
			\end{tabular}\par}%
		\vskip 2em%
		{\large Hochschule Bonn-Rhein-Sieg, Fachbereich Informatik, D-53757\par}%
		\vskip 2em%
		{\large \@date}%
	\end{center}%
	\par
	\vskip 1.5em}
\makeatother


% Commands
\newcommand{\sectionspace}{
	\vspace{0.5cm}
}




\begin{document}
	
\begin{titlepage}
	\maketitle
\end{titlepage}


\tableofcontents
\newpage

\section{Hintergrund}\label{sec:Hintergrund}

\sectionspace
\section{Formale Elemente}\label{sec:Formale_Elemente}

\sectionspace
\section{Spielende (Zielgruppe)}\label{sec:Spieler}
Die spielende Person spielt einen ins Spielgeschehen aktiv eingebundenen Avatar, mit der er/sie zunächst überwiegend bauen, im späteren Spielverlauf zum großen Teil managen, aber immer aktiv mit der Spielwelt interagieren soll. Die Top-Down Perspektive mit dem Avatar im Mittelpunkt soll der spielenden Person die nötige nähe des Charakters vermitteln, dennoch genügend Abstand bieten, um das Umfeld im Blick zu behalten. Das Spiel wird ausschließlich im Einzelspieler-Modus spielbar sein, in welchem die spielende Person gegen Herausforderungen des Spiels antritt. Je nach eingestelltem Schwierigkeitsgrad sind diese leichter bzw. schwieriger und treten seltener bzw. häufiger auf. \\
Da es sich bei diesem Spiel um ein Aufbau-Strategiespiel handelt, welches recht friedlich, in gewissen Situationen dennoch anspruchsvoll ist, eignet sich das Spiel nicht für jede Altersgruppe, kann aber auch von jungen Spielenden verstanden und gespielt werden. Die in Deutschland für die Altersfreigabe von Videospielen verantwortliche Stelle ``Unterhaltungssoftware Selbstkontrolle'' (kurz USK) stuft Aufbau-Strategiespiele meistens unter USK ab 6 Jahren \cite{usk_6} ein, was auch für dieses Spiel eine angemessene Altersgrenze darstellt. Das Spiel eignet sich ebenso für ältere Spielende, wobei hier beachtet werden muss, dass bestimmte mentale Fähigkeiten gefordert sind, die bei der spielenden Person vorhanden sein müssen.

\sectionspace
\section{Gegenstand \& Ziel des Spiels}\label{sec:Gegenstand}

\sectionspace
\section{Abläufe}\label{sec:Ablaeufe}

\sectionspace
\section{Regeln}\label{sec:Regeln}

\sectionspace
\section{Ressourcen}\label{sec:Ressourcen}

\sectionspace
\section{(potentielle) Konflikte, absehbare Schwierigkeiten}\label{sec:Konflikte}

\sectionspace
\section{Rahmenbedingungen}\label{sec:Rahmenbedingungen}

\sectionspace
\section{Ergebnisse}\label{sec:Ergebnisse}

\sectionspace
\printbibliography[heading=bibnumbered, title=Referenzen]\label{sec:Referenzen}

\end{document}
