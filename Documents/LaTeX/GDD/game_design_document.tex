\documentclass[paper=A4,pagesize=auto,12pt,headinclude=true,footinclude=true,BCOR=0mm,DIV=calc]{scrartcl}

\usepackage[T1]{fontenc}
\usepackage[utf8]{inputenc}
\usepackage[ngerman]{babel}
\usepackage{lmodern}
\usepackage[hidelinks]{hyperref}
\usepackage{setspace}
\usepackage[left=25mm, right=25mm, top=25mm, bottom=25mm]{geometry}
\usepackage[backend=biber, style=ieee, citestyle=ieee]{biblatex}
\usepackage[nottoc,numbib]{tocbibind}
\onehalfspacing

\bibliography{game_design_document}

% Title
\title{Game Design Document}

\author{Alexandra Zarkh, Sui Yin Zhang, Lennart Leggewie,\\ Stefan Glorch, Alexander Schallenberg}


\makeatletter
\def\@maketitle{%
	\newpage
	\null
	\vskip 2em%
	\begin{center}%
		\let \footnote \thanks
		{\Huge \textbf{\@title} \par}%
		\vskip 1.5em%
		{\Large Game Development ``Städtebau''\par}%
		\vskip 3em%
		{\large
			\lineskip .5em%
			\begin{tabular}[t]{c}%
				\@author
			\end{tabular}\par}%
		\vskip 2em%
		{\large Hochschule Bonn-Rhein-Sieg, Fachbereich Informatik, D-53757\par}%
		\vskip 2em%
		{\large \@date}%
	\end{center}%
	\par
	\vskip 1.5em}
\makeatother


% Commands
\newcommand{\sectionspace}{
	\vspace{0.5cm}
}




\begin{document}
	
\begin{titlepage}
	\maketitle
\end{titlepage}


\tableofcontents
\newpage

\section{Spieler (Zielgruppe)}\label{sec:Spieler}
Die Unterhaltungssoftware Selbstkontrolle stuft Aufbau-Strategiespiele standardmäßig unter USK ab 6 Jahren \cite{usk_6} ein.

\sectionspace
\section{Gegenstand \& Ziel des Spiels}\label{sec:Gegenstand}

\sectionspace
\section{Abläufe}\label{sec:Ablaeufe}

\sectionspace
\section{Regeln}\label{sec:Regeln}

\sectionspace
\section{Ressourcen}\label{sec:Ressourcen}

\sectionspace
\section{(potentielle) Konflikte, absehbare Schwierigkeiten}\label{sec:Konflikte}

\sectionspace
\section{Rahmenbedingungen}\label{sec:Rahmenbedingungen}

\sectionspace
\section{Ergebnisse}\label{sec:Ergebnisse}

\sectionspace
\section{Referenzen}\label{sec:Referenzen}

\newpage
\printbibliography[heading=bibnumbered]

\end{document}
